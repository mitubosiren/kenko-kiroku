\documentclass{jsarticle}
\usepackage[dvipdfmx]{graphicx}
%\usepackage{fancyhdr}
\usepackage{fancybox}
\usepackage {colortbl,array,xcolor}
\pagestyle{empty}
\newcommand{\ctext}[1]{\raise0.2ex\hbox{\textcircled{\scriptsize{#1}}}}
\title{健康観察記録表}

\author{\fbox{学籍番号 20XXXXXXX 所属 〇〇学群〇〇学類 氏名 筑波太郎}\and \fbox{平熱 . 度}}
\date{}
\begin{document}
%\thispagestyle{fancy}
%\rhead{\Large{筑波大学}}
%\cfoot{}
%\renewcommand{\headrulewidth}{0.0pt}
\maketitle
\begin{enumerate}
\item{毎朝、起床時と夕方検温し、下の表に記録してください。}
\item{その他、風邪の症状の有無(有りの場合は具体的症状)に◯をつけてください。}
\end{enumerate}

\begin{table}[htb]
\begin{tabular}{|c|c|c|c|c|c|}\hline
日付&曜&$\ $朝の体温$\ $&夕方の体温&風邪症状&備考\\ \hline
\rowcolor[gray]{0.8}
例&月&36.5度&36.6度&\ctext{無}・有(咳・鼻水・くしゃみ・咽頭痛・倦怠感・呼吸困難感)&\\ \hline
月日&&度&度&無・有(咳・鼻水・くしゃみ・咽頭痛・倦怠感・呼吸困難感)&\\ \hline
月日&&度&度&無・有(咳・鼻水・くしゃみ・咽頭痛・倦怠感・呼吸困難感)&\\ \hline
月日&&度&度&無・有(咳・鼻水・くしゃみ・咽頭痛・倦怠感・呼吸困難感)&\\ \hline
月日&&度&度&無・有(咳・鼻水・くしゃみ・咽頭痛・倦怠感・呼吸困難感)&\\ \hline
月日&&度&度&無・有(咳・鼻水・くしゃみ・咽頭痛・倦怠感・呼吸困難感)&\\ \hline
月日&&度&度&無・有(咳・鼻水・くしゃみ・咽頭痛・倦怠感・呼吸困難感)&\\ \hline
月日&&度&度&無・有(咳・鼻水・くしゃみ・咽頭痛・倦怠感・呼吸困難感)&\\ \hline
月日&&度&度&無・有(咳・鼻水・くしゃみ・咽頭痛・倦怠感・呼吸困難感)&\\ \hline
月日&&度&度&無・有(咳・鼻水・くしゃみ・咽頭痛・倦怠感・呼吸困難感)&\\ \hline
月日&&度&度&無・有(咳・鼻水・くしゃみ・咽頭痛・倦怠感・呼吸困難感)&\\ \hline
月日&&度&度&無・有(咳・鼻水・くしゃみ・咽頭痛・倦怠感・呼吸困難感)&\\ \hline
月日&&度&度&無・有(咳・鼻水・くしゃみ・咽頭痛・倦怠感・呼吸困難感)&\\ \hline
月日&&度&度&無・有(咳・鼻水・くしゃみ・咽頭痛・倦怠感・呼吸困難感)&\\ \hline
月日&&度&度&無・有(咳・鼻水・くしゃみ・咽頭痛・倦怠感・呼吸困難感)&\\ \hline
\end{tabular}
\end{table}


\end{document}
